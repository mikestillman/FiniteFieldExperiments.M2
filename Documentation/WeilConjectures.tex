\documentclass[12pt,a4paper]{amsart}       
\usepackage{german,amsmath,amssymb,latexsym,amsfonts}   
\usepackage{graphicx}              %                
\usepackage{enumerate}
%\setlength{\topmargin}{-1cm} 
%\setlength{\textheight}{23.5cm} 
%\setlength{\textwidth}{20cm}
%\setlength{\evensidemargin}{-4mm}  
%\setlength{\oddsidemargin}{-4mm}  

%\setlength{\mathsurround}{2pt}
%\setlength{\parindent}{0mm}
%\setlength{\parskip}{3mm}


%%%%%%%%%%%%%%%%%%%%%%%%%%%%%%%%%%%%%%%%%%%%%%%%%%%%%%%%%%%%%%%%%%%%
%%%%% for M2 code %%%%%%%%%%%%%%%%%%%%%%%%%%%%%%%%%%%%%%%%%%%%%%%%%%
%%%%%%%%%%%%%%%%%%%%%%%%%%%%%%%%%%%%%%%%%%%%%%%%%%%%%%%%%%%%%%%%%%%%
%% use this verbatim like environment to insert Macaulay2 code.
%% Another program can replace this block with it execution
\usepackage{fancyvrb}

\newenvironment{m2}
  {
    % First, we tell fancyvrb that we're inside a verbatim environment.
    \tiny
    \VerbatimEnvironment
    \begin{Verbatim}[frame=single]%
  }
  {\end{Verbatim}}
%%%%%%%%%%%%%%%%%%%%%%%%%%%%%%%%%%%%%%%%%%%%%%%%%%%%%%%%%%%%%%%%%%%%

\usepackage[left=1.5in,top=1in,right=1.5in,bottom=1in]{geometry}

\usepackage{graphicx}
\usepackage{color}

\usepackage[all]{xy}

 \renewcommand{\thesubsection}{\thesection.\alph{subsection}}

\theoremstyle{plain}
\newtheorem{theorem}{Theorem}[section]
\newtheorem*{expectation}{Expectation}
\newtheorem{proposition}[theorem]{Proposition}
\newtheorem{lemma}[theorem]{Lemma}
\newtheorem{corollary}[theorem]{Corollary}
\newtheorem{problem}[theorem]{Problem}


% Separate numbering for Theorem statements in the introduction
\newtheorem{theoremintro}{Theorem}
\newcommand{\theHtheoremintro}{\Alph{theoremintro}}

\theoremstyle{definition}
\newtheorem{definition}[theorem]{Definition}
\newtheorem{example}[theorem]{Example}
\newtheorem{notation}[theorem]{Notation}

%\theoremstyle{remark}
\newtheorem{remark}[theorem]{Remark}


\newcommand{\residue}{\partial}
\newcommand{\divisor}{\mathrm{div}}
\newcommand{\Brtwo}{{}_2\mathrm{Br}}
\newcommand{\Pictwo}{{}_2\mathrm{Pic}}

\newcommand{\isom}{\cong}
\newcommand{\isomto}{\simto}
\newcommand{\isometry}{\cong}

\newcommand{\FF}{\mathbb F}
\newcommand{\ZZ}{\mathbb Z}
\renewcommand{\AA}{\mathbb A}
\newcommand{\CC}{\mathbb C}
\newcommand{\PP}{\mathbb P}
\newcommand{\QQ}{\mathbb Q}
\newcommand{\Gm}{\mathbb{G}_{\mathrm{m}}}
%\newcommand{\T}{\mathbb{T}}

\DeclareMathOperator{\AAut}{\mathbf{Aut}}
\DeclareMathOperator{\Aut}{\mathrm{Aut}}
\DeclareMathOperator{\Br}{\mathrm{Br}}
\DeclareMathOperator{\Isom}{\mathrm{Isom}}
\DeclareMathOperator{\IIsom}{\Group{Isom}}
\DeclareMathOperator{\PPic}{\sheaf{P}\!\mathit{ic}}
\DeclareMathOperator{\Pic}{\mathrm{Pic}}
\DeclareMathOperator{\rk}{\mathrm{rk}}
\DeclareMathOperator{\SSpec}{\mathbf{Spec}}
\DeclareMathOperator{\Spec}{\mathrm{Spec}}
\DeclareMathOperator{\EExt}{\sheaf{E}\!\mathit{xt}}
\DeclareMathOperator{\Ext}{\mathrm{Ext}}
\DeclareMathOperator{\coker}{\mathrm{coker}}
\DeclareMathOperator{\Mor}{Mor}
\DeclareMathOperator{\codim}{codim}
%\DeclareMathOperator{\log}{log}

\newcommand{\inv}{^{-1}}
\newcommand{\sep}{^{\mathrm{s}}}
\newcommand{\mult}{^{\times}}
\newcommand{\dual}{^{\vee}}
\newcommand{\tensor}{\otimes}
\newcommand{\bslash}{\smallsetminus}
\newcommand{\mapto}[1]{\xrightarrow{#1}}
\newcommand{\ol}[1]{\overline{#1}}
\newcommand{\ul}[1]{\underline{#1}}
\newcommand{\wt}[1]{\widetilde{#1}}
\newcommand{\et}{\mathrm{\acute{e}t}}

\newcommand{\linedef}[1]{\textsl{#1}}
\newcommand{\Het}{H_{\et}}
\newcommand{\ur}{\mathrm{nr}}
%\newcommand{\ur}{\mathrm{ur}}
\newcommand{\Hur}{H_{\ur}}
\newcommand{\merk}{\mathrm{r}}
\newcommand{\Hr}{H_{\merk}}
\newcommand{\Pfister}[1]{\ll\!{#1}\gg}
\newcommand{\quadform}[1]{<\! #1 \!>}
\newcommand{\Local}{\mathsf{Local}}
\newcommand{\Ab}{\mathsf{Ab}}
\newcommand{\Var}{\mathsf{Var}}
\newcommand{\Frac}{\mathrm{Frac}}
\newcommand{\im}{\mathrm{im}}
\newcommand{\CH}{\mathrm{CH}}
\newcommand{\CM}{\mathsf{CM}}
\newcommand{\res}{\mathrm{res}}
\newcommand{\cores}{\mathrm{cor}}
\newcommand{\ord}{\mathrm{ord}}
\newcommand{\Norm}{\mathrm{N}}
\newcommand{\vp}{\varphi}
\newcommand{\Hom}{\mathrm{Hom}}
\newcommand{\id}{\mathrm{id}}

\newcommand\WHY{{\color{red}\textsf{WHY?}}~}

%\renewcommand{\labelenumi}{\it\alph{enumi})}
\renewcommand\theenumi{\it\alph{enumi}}
\renewcommand\labelenumi{\theenumi)}

\usepackage[backref=page]{hyperref}

%% This is supposed to be translated into "@TO....@" in the Macaulay2 documentation
\newcommand{\link}[1]{\tt #1}

\begin{document}

\title{Finite Field Experiments}

\author{Jakob Kr\"oker}
\author{Mike Stillmann}
\author{Hans-Christian v.\,Bothmer}

\maketitle

\section{Weil Conjectures}

We follow Hartshorne Appendix C. 
\medskip

Let $X$ projective variety of dimension $n$ defined over $\ZZ$. Assume $X$ is smooth over $\FF_q$, $q$ a prime number. It is then automatically smooth over $\CC$. 

Let now
\[
	N_r := \left| X\bigl(\FF_{q^r}\bigr) \right|
\]
the number of rational points of $X$ over $\FF_{q^r}$. Then 
\[
	Z(t) := \exp \left( \sum_{r=1}^{\infty} N_r \frac{t^r}{r} \right)
\]
is called {\sl zeta function} of $X$. Notice that
\[
	Z(t) = 1 + N_1 t + \dots
\]
i.e. that the linear term of $Z(t)$ is the number of rational points of $X$ over $\FF_q$. Now from the Weil conjectures we now, that there are polynomials $P_i \in \ZZ[t]$ such that
\begin{enumerate}
\item $Z(t) = \frac{P_1(t) P_3(t) \cdots P_{2n-1}(t)}{P_0(t)P_2(t) \cdots P_{2n}(t)}$
\item $P_0(t) = 1 - t$
\item $P_{2n}(t) = 1 - q^n t$
\item $P_{i}(t) = \prod (1 - \alpha_{ij} t)$ with $\alpha_{ij}$ algebraic integers with $|\alpha_{ij}| = \sqrt{q^i}$.
\item $\deg P_{i} = B_i = h^i(X_{\CC},\ZZ)$, where $X_\CC$ is the variety defined by the equations of $X$ over $\CC$.
\end{enumerate}
From this information we can compute the linear term of the zeta function as
\[
	Z(t) = 1 + \left( 1 + q^n+ \sum_{\text{$i$ even}} \alpha_{ij} - \sum_{\text{$i$ odd}} \alpha_{ij} \right) t + \dots
\]
Now observe that
\[
	\left| \sum_{\text{$i$ even}} \alpha_{ij} - \sum_{\text{$i$ odd}} \alpha_{ij} \right| 
	\le \sum_{\text{$i$ even}} |\alpha_{ij}| + \sum_{\text{$i$ odd}} |\alpha_{ij}|  
	= \sum_{i=1}^{2n-1} B_i \sqrt{q^i}
\]
and therefore
\[
	| N_1 - (1+q^n) | \le \sum_{i=1}^{2n-1} B_i \sqrt{q^i}
\]
Notice that if one thinks of $N_1$ as a random variable and of the $\alpha_{ij}$ as uniformly distributed on the circle with radius $\sqrt{q^i}$ then the distribution of $N_1$ would have expected value
\[
	\mu(N_1) = 1+ q^n
\]
and standard deviation
\[
	\sigma(N_1) = \sqrt{\sum_{i=1}^{2n-1} B_i q^i}.
\]


\end{document}
